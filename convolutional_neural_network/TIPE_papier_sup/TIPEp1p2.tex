\documentclass[a4paper,11pt]{article}
\usepackage[latin1]{inputenc}
\usepackage[T1]{fontenc}
\usepackage[francais]{babel}
\usepackage{amsmath}
\usepackage{amssymb}
\usepackage{graphicx}
\usepackage{lmodern}
\usepackage{textcomp}
\usepackage[a4paper]{geometry}
\usepackage{xcolor}
\usepackage{listings}
\title{TIPE: L'oc\'{e}an}
\date{2019}

% Definition de couleur supplementaire
\definecolor{colString}{rgb}{0.6,0.1,0.1}
 
% Definition du langage
\lstdefinelanguage{LangageConsole}{%
    morekeywords={%
        ligne% mot-cl\'{e} ``ligne''
    }
}
 
% Definition du style
\lstdefinestyle{styleLangage}{%
    language        = LangageConsole,%
    basicstyle      = \footnotesize\ttfamily\color{black},% ecriture standard
    identifierstyle = \color{black},%
    commentstyle    = \color{orange},%
    keywordstyle    = \color{blue},%
    stringstyle     = \color{colString},%
    extendedchars   = true,% permet d'avoir des accents dans le code
    tabsize         = 2,%
    showspaces      = false,%
    showstringspaces = false,%
    numbers=left,%
    numberstyle=\tiny\ttfamily\color{black},%
    breaklines=true,%
    breakautoindent=true,%
        backgroundcolor=\color{white},%
}
 
\lstset{%
    style = styleLangage%
}



\begin{document}
\maketitle
\paragraph{Cassiop\'{e}e WILLAUME-C\'{E}GLAREK, Damien DUVERT, Guillaume HUGON DE MASGONTIER \newline \newline\newline \newline\newline\newline \newline}
\begin{center}
\huge{D\'{e}tection d'obstacles et d'objets en navigation marine et sous-marine} \newpage
\end{center}
\part*{Introduction}
\normalsize \paragraph{\normalfont \textbf{"Les moindres bruits se transmettaient avec une vitesse \`{a} laquelle l'oreille n'est pas habitu\'{e}e sur la terre. En effet, l'eau est pour le son un meilleur v\'{e}hicule que l'air, et il s'y propage avec une rapidit\'{e} quadruple. (...)  Des bruits profonds, nettement transmis par ce milieu liquide, se r\'{e}percutaient avec une majestueuse ampleur."} \newline Jules Verne, \textit{Vingt Mille Lieues sous les Mers.} \newline \newline
L'acoustique sous-marine est apparue au d\'{e}but du XX\up{\`{e}me} si\`{e}cle avec les exp\'{e}riences de Paul Langevin, et a prouv\'{e} de mani\`{e}re d\'{e}finitive son importance sur le plan militaire durant la Seconde Guerre Mondiale. Depuis, les techniques de d\'{e}tection acoustique ont consid\'{e}rablement \'{e}volu\'{e}. En outre, les applications \`{a} l'oc\'{e}anographie, la prospection et l'exploitation du p\'{e}trole offshore se d\'{e}veloppent, constituant le moyen privil\'{e}gi\'{e} d'exploration du monde marin \`{a} grande distance. \newline \newline
Le domaine sous-marin \'{e}chappe presque totalement \`{a} l'utilisation des ondes \'{e}lectromagn\'{e}tiques et lumineuses car le caract\`{e}re dissipatif de l'eau sal\'{e}e (li\'{e} \`{a} sa forte conductivit\'{e}) entra\^{i}ne une att\'{e}nuation tr\`{e}s importante. Les ondes acoustiques, qui sont des vibrations m\'{e}caniques du milieu de propagation, constituent alors le seul vecteur permettant de transporter des informations exploitables sous la mer. De plus, les ondes sonores poss\`{e}dent de meilleures caract\'{e}ristiques de transmission dans l'eau que dans l'air : leur vitesse de propagation est plus \'{e}lev\'{e}e et, puisqu'elles subissent moins d'att\'{e}nuation, elles peuvent se propager \`{a} des distances plus grandes. Cela justifie l'utilisation des ondes acoustiques dans le domaine sous-marin, m\^{e}me si les signaux sont perturb\'{e}s par le bruit ambiant et les \'{e}chos parasites.
L'acoustique sous-marine permet aujourd'hui d'assurer les m\^{e}mes fonctions que les ondes radar et radio dans l'atmosph\`{e}re et l'espace (m\^{e}me si leurs performances sont souvent inf\'{e}rieures). En effet, on l'utilise pour mesurer les caract\'{e}ristiques du milieu marin, transmettre des signaux, et d\'{e}tecter et localiser des obstacles ou des cibles (fonction premi\`{e}re des syst\`{e}mes sonar), ce qui fera l'objet de notre TIPE.\newline}
\begin{center}
\large{\underline{Probl\'{e}matique:}}
\end{center}
\paragraph{Comment caract\'{e}riser les ondes acoustiques maritimes et les utiliser dans le cadre de la d\'{e}tection marine d'objets?} \newpage
\part{Caract\'{e}risation de la propagation des ondes acoustiques maritimes et influence des param\`{e}tres du milieu}
\section{G\'{e}n\'{e}ralit\'{e}s sur les ondes acoustiques - Caract\'{e}risation de la propagation}
\paragraph{\normalfont Les ondes acoustiques sont provoqu\'{e}es par la propagation d'une perturbation m\'{e}canique (transmission d'une s\'{e}rie de compressions/dilatations locales d'un \'{e}l\'{e}ment du milieu \`{a} un \'{e}l\'{e}ment adjacent), et n\'{e}cessitent donc pour se propager un support mat\'{e}riel \'{e}lastique, dont les caract\'{e}ristiques m\'{e}caniques d\'{e}terminent la c\'{e}l\'{e}rit\'{e} des ondes. L'onde acoustique est ainsi caract\'{e}ris\'{e}e par l'\textbf{amplitude du mouvement local} de chaque \'{e}l\'{e}ment autour de sa position d'\'{e}quilibre et la \textbf{vitesse particulaire correspondante}, ainsi que par la \textbf{pression acoustique} (variation autour de la pression statique moyenne) qui en r\'{e}sulte. La c\'{e}l\'{e}rit\'{e} de l'onde acoustique d\'{e}pend de la masse volumique $\rho$ et du module d'\'{e}lasticit\'{e} E du milieu selon la relation $c=\sqrt{E/\rho}$, avoisinant les 1500 m.s\up{-1} dans l'eau sal\'{e}e. \newline \newline
\underline{\'{E}quation d'onde :}\newline
Les ph\'{e}nom\`{e}nes ondulatoires d\'{e}crivent la propagation de perturbations \`{a} travers l'espace en fonction du temps. Pour les ondes se propageant dans une seule dimension spatiale $x$, les variations d'une grandeur physique en jeu dans le ph\'{e}nom\`{e}ne ondulatoire sont d\'{e}crites par une fonction $f(x,t)$ de l'espace et du temps. On prend pour \textbf{convention $f(x,t) = 0$ en l'absence de perturbation}. La propagation de l'onde est alors d\'{e}crite par l'\textbf{\underline{\'{e}quation de d'Alembert \`{a} une dimension}} :}
\[ \frac{1}{c^{2}}\frac{\partial^{2} f(x,t)}{\partial t^{2}}-\frac{\partial^{2} f(x,t)}{\partial x^{2}}=0\]
\paragraph{\normalfont L'\'{e}quation de d'Alembert est exacte dans le cas des ondes \'{e}lectromagn\'{e}tiques dans le vide mais n'est valable que pour de petites perturbations dans le cas des ondes m\'{e}caniques puisqu'on fait l'hypoth\`{e}se que les perturbations se propagent \textbf{sans att\'{e}nuation et sans dispersion}.
Dans le cas d'une onde acoustique, l'onde est caract\'{e}ris\'{e}e \textbf{indiff\'{e}remment} par le d\'{e}placement des \'{e}l\'{e}ments de fluide ou par la variation de pression autour de la pression moyenne. On montre qu'une onde sonore de faible amplitude \`{a} une dimension dans un fluide au repos est caract\'{e}ris\'{e}e par les trois \'{e}quations fondamentales suivantes (principe fondamental de la dynamique, conservation de la masse, compressibilit\'{e}) : }
\begin{equation}
\frac{\partial v}{\partial t}=-\frac{1}{\rho_{0}}\frac{\partial p}{\partial x}
\end{equation}
\begin{equation}
\frac{\partial \delta \rho}{\partial t}=-\rho_{0}\frac{\partial v}{\partial x}
\end{equation}
\begin{equation}
\frac{\partial p}{\partial t}=-\frac{1}{\chi_{s}}\frac{\partial v}{\partial x}
\end{equation}
\paragraph{\normalfont o\`{u} $v$ est la vitesse locale, t le temps, p la pression acoustique locale, x la position, $\rho$ la masse volumique locale, $\rho_{0}$ la masse volumique pour le fluide au repos, $\chi_s$  la compressibilit\'{e} du fluide. \newline\newline
En d\'{e}rivant  une fois par rapport au temps l'\'{e}quation (3) au premier ordre dans le champ de vitesse, on obtient la relation:}
\[ \frac{\partial^{2} p(x,t)}{\partial t^{2}}= -\frac{1}{\chi_{s}}\frac{\partial}{\partial x} (\frac{\partial v(x,t)}{\partial t})\]
\paragraph{\normalfont En utilisant l'\'{e}quation du mouvement (1), on obtient alors:}
\[ \frac{\partial^{2} p(x,t)}{\partial t^{2}}= -\frac{1}{\chi_{s}}\frac{\partial}{\partial x} (-\frac{1}{\rho_0}\frac{\partial p(x,t)}{\partial x})\]
\paragraph{\normalfont soit finalement:}
\[ \rho_0\chi_s \frac{\partial^{2} p(x,t)}{\partial t^{2}}-\frac{\partial^2 p(x,t)}{\partial x^2}=0\]
\paragraph{\normalfont En identifiant la c\'{e}l\'{e}rit\'{e} des ondes sonores comme: $c=1/\sqrt{\rho_0\chi_s}$, on peut r\'{e}\'{e}crire l'\'{e}quation pr\'{e}c\'{e}dente comme suit :}
\[ \frac{1}{c^{2}}\frac{\partial^{2} p(x,t)}{\partial t^{2}}-\frac{\partial^{2} p(x,t)}{\partial x^{2}}=0\]
\paragraph{\normalfont On obtient aussi que la vitesse $v(x,t)$ satisfait l'\'{e}quation d'onde:}
\[ \frac{1}{c^{2}}\frac{\partial^{2} v(x,t)}{\partial t^{2}}-\frac{\partial^{2} v(x,t)}{\partial x^{2}}=0\]
\paragraph{\normalfont Une cons\'{e}quence des \'{e}quations (1), (2) et (3) est la relation :}
\[\partial_xv=-\chi_s\partial_tp=-\frac{1}{\rho_0}\]
\paragraph{\normalfont soit encore $\partial_t\delta p=\rho_0\chi_s\partial_tp$. D'o\`{u} $p(x,t)=c^2\delta p(x,t)+\alpha(x)$. En imposant que l'onde s'annule pour $t\rightarrow +\infty$, on a $\alpha(x) = 0$. \newline
En \textbf{\underline{conclusion}}, on peut d\'{e}finir la vitesse du son par la relation:}
\[\boxed{p(x,t)=c^2\delta p(x,t)}\]
\paragraph{\normalfont Int\'{e}ressons-nous maintenant aux trois dimensions de l'espace. Les ondes acoustiques maritimes ob\'{e}issent aux lois de la m\'{e}canique des fluides. On montre que leur propagation est d\'{e}crite de mani\`{e}re g\'{e}n\'{e}rale par l'\textbf{\underline{\'{e}quation}} \textbf{\underline{de Helmholtz}}:}
\[\Delta p=\frac{\partial^2p}{\partial x^2}+\frac{\partial^2p}{\partial y^2}+\frac{\partial^2p}{\partial z^2}=\frac{1}{c^2(x,y,z)}\frac{\partial^2p}{\partial t^2}\]
\paragraph{\normalfont o\`{u} $p$ est la pression d'une onde se d\'{e}pla\c{c}ant au cours du temps $t$ dans l'espace $(x,y,z)$ et $c(x,y,z)$ la c\'{e}l\'{e}rit\'{e} locale de propagation de l'onde. Si la c\'{e}l\'{e}rit\'{e} est restreinte \`{a} la seule direction $x$, on retrouve l'expression \'{e}tablie plus haut:}
\[ \frac{1}{c^{2}}\frac{\partial^{2} p}{\partial t^{2}}=\frac{\partial^{2} p}{\partial x^{2}}\]
\paragraph{\normalfont La vitesse particulaire $v(t)$ est obtenue \`{a} partir de la pression par $p=\rho cv$, o\`{u} $\rho$ est la masse volumique. Le produit $\rho c$, qui traduit le rapport entre vitesse particulaire et pression acoustique, est appell\'{e} imp\'{e}dance caract\'{e}ristique du milieu de propagation. \newline \newline
\underline{Particularit\'{e}s et d\'{e}fauts de la propagation des ondes acoustiques en milieu} \newline \underline{sous-marin :}\newline
L'eau de mer est un milieu tr\`{e}s favorable \`{a} la propagation des ondes acoustiques, mais pr\'{e}sente tout de m\^{e}me de nombreuses limitations. Les signaux sont \textbf{att\'{e}nu\'{e}s} et leur port\'{e}e limit\'{e}e \`{a} cause de l'\textbf{absorption des ondes dans l'eau}. Les \textbf{variations de c\'{e}l\'{e}rit\'{e}} du son dues \`{a} l'\textbf{inhomog\'{e}n\'{e}it\'{e} du milieu}, la \textbf{r\'{e}flexion des signaux sur les interfaces} et les \textbf{interf\'{e}rences entre signaux} perturbent la propagation, qui ne se fait pas toujours en ligne droite. En outre, les mouvements des sonars et des cibles font intervenir l'effet Doppler, et le \textbf{bruit ambiant de l'oc\'{e}an} ajout\'{e} au bruit propre de l'engin sous-marin masquent le signal utile. \newline}
\paragraph{\normalfont Le ph\'{e}nom\`{e}ne de r\'{e}flexion est omnipr\'{e}sent dans le domaine sous-marin. En effet, le fond et la surface de l'eau sont des interfaces sur lesquelles les ondes sonores se r\'{e}fl\'{e}chissent en permanence. Par cons\'{e}quent, les \'{e}chos se multiplient et parasitent la r\'{e}ception. Plus le nombre de r\'{e}flexions est \'{e}lev\'{e}, plus la baisse d'intensit\'{e} acoustique est \'{e}lev\'{e}e et plus le temps de trajet est grand. Il faut donc privil\'{e}gier le signal empruntant le trajet direct (\textbf{sous r\'{e}serve d'existence de ce trajet}) car il est le plus rapide et le moins att\'{e}nu\'{e}.\newline}
\paragraph{\normalfont Par ailleurs, les ondes acoustiques sont amorties selon des lois qui d\'{e}pendent de leur fr\'{e}quence et des caract\'{e}ristiques du milieu, qui est dissipatif et absorbe une partie de l'\'{e}nergie de l'onde \`{a} cause de sa viscosit\'{e}. Ainsi, le ph\'{e}nom\`{e}ne de \textbf{relaxation ionique} appara\^{i}t. La relaxation ionique est la \textbf{dissociation de certains compos\'{e}s ioniques en solution du fait de la variation locale de pression cr\'{e}\'{e}e par le passage de l'onde acoustique}. Si la p\'{e}riode est sup\'{e}rieure au temps n\'{e}cessaire pour la mol\'{e}cule pour se recomposer (le temps de relaxation), le ph\'{e}nom\`{e}ne se reproduit \`{a} chaque alternance et \textbf{dissipe de l'\'{e}nergie en permanence}. L'att\'{e}nuation apparait donc pour des fr\'{e}quences inf\'{e}rieures \`{a} la fr\'{e}quence de relaxation caract\'{e}ristique du compos\'{e} concern\'{e}. Il y a relaxation des mol\'{e}cules de sulfate de magn\'{e}sium MgSO$_4^-$ au-del\`{a} de 100kHz et relaxation de l'acide borique $B(OH)_3$ au de la de 1kHz.\newline \newline
On d\'{e}finit alors un coefficient d'amortissement, qui \'{e}volue fortement avec la fr\'{e}quence. Ci-dessous une repr\'{e}sentation de l'amortissement des ondes sonores dans l'eau de mer en fonction de leur fr\'{e}quence, tabul\'{e} \`{a} plusieurs temp\'{e}ratures et pour une salinit\'{e} de 35\textperthousand.}
\begin{figure}[!h] %on ouvre l'environnement figure
\includegraphics[scale=0.80]{TIPE_FIG1.png} %ou image.png, .jpeg etc.
\caption{Repr\'{e}sentation de la mutiplicit\'{e} des trajets pour une onde sonore} %la l\'{e}gende
\label{fig_1} %l'\'{e}tiquette pour faire r\'{e}f\'{e}rence \`{a} cette image
\end{figure}
\paragraph{\normalfont Ainsi, \textbf{plus la fr\'{e}quence est \'{e}lev\'{e}e, plus l'amortissement est important}. C'est pourquoi les syst\`{e}mes de sonar privil\'{e}gient des \textbf{fr\'{e}quences relativement basses}, aux alentours de quelques dizaines de kHz. En effet, si une fr\'{e}quence de 10 kHz permet d'atteindre des distances de quelques dizaines de kilom\`{e}tres (le coefficient d'amortissement est de l'ordre de 1dB/km), une fr\'{e}quence de 1 MHz limite les syst\`{e}mes de d\'{e}tection \`{a} moins de 100 m de port\'{e}e (le coefficient d'amortissement est proche de 500dB/km). \newline \newline \`{A} cette absorption de l'\'{e}nergie acoustique par le milieu de propagation s'ajoutent les \textbf{pertes} (notamment en termes d'intensit\'{e} de l'onde sonore) s'expliquant par le ph\'{e}nom\`{e}ne de divergence g\'{e}om\'{e}trique. Durant la propagation de l'onde, il y a \textbf{conservation de l'\'{e}nergie acoustique \'{e}mise}. N\'{e}anmoins, comme la surface sur laquelle se r\'{e}partit cette \'{e}nergie augmente au fur et \`{a} mesure de la propagation de l'onde, \textbf{l'intensit\'{e} acoustique diminue proportionnellement \`{a} l'inverse de cette surface}. C'est ce qu'on appelle les pertes par \textbf{\underline{divergence g\'{e}om\'{e}trique}}. Par exemple, dans le cas d'un milieu infini homog\`{e}ne et d'une source omnidirectionnelle de faibles dimensions, on observe une \textbf{divergence sph\'{e}rique} (les surfaces occup\'{e}es sont des surfaces de sph\`{e}res de rayons croissants). \newline \newline
Afin de pr\'{e}voir les pertes de propagation, un calcul pour une dispersion sph\'{e}rique permet d'estimer le niveau de pertes en d\'{e}cibels selon la relation: $PT = 20 log(R) + \alpha R$ o\`{u} R est la distance (en km) et $ \alpha $ le coefficient d'amortissement (en dB.km\up{-1}) \newline \newline
Enfin, la \textbf{r\'{e}verb\'{e}ration acoustique} participe \`{a} la \textbf{d\'{e}formation du signal \'{e}mis}. Une partie de l'\'{e}nergie acoustique se propage vers les r\'{e}cepteurs en suivant d'autres chemins que les rayons propres, avec des temps de propagation g\'{e}n\'{e}ralement plus \'{e}lev\'{e}s. On distingue les r\'{e}verb\'{e}rations "de surface", "de fond" et "de volume" selon le chemin emprunt\'{e} par ces trains d'ondes.  On observe alors un signal large bande dans la repr\'{e}sentation temps-fr\'{e}quence des signaux les plus \'{e}nerg\'{e}tiques, ce qui peut perturber l'analyse.}
\begin{figure}[!h]
\includegraphics[scale=0.90]{TIPE_FIG2.png}
\caption{Amortissement du son dans l'eau de mer en fonction de la fr\'{e}quence}
\label{fig_2}
\end{figure}
\section{Influence des caract\'{e}ristiques du milieu - R\'{e}sultats exp\'{e}rimentaux}
\paragraph{\normalfont Les caract\'{e}ristiques du milieu varient dans l'espace et dans le temps. Les signaux acoustiques fluctuent alors en fonction de la temp\'{e}rature, la salinit\'{e}, le relief du fond, la houle, les courants, ainsi que les mouvements relatifs des \'{e}metteurs et des cible. La c\'{e}l\'{e}rit\'{e} du son varie dans le m\^{e}me sens que la temp\'{e}rature, la salinit\'{e} et l'immersion. La salinit\'{e} correspond au pourcentage massique de sels dissous dans l'eau pure. Elle est exprim\'{e}e en \textperthousand, et sa valeur moyenne est d'environ 35\textperthousand. La temp\'{e}rature d\'{e}cro\^{i}t globalement de la surface vers le fond. Ses variations spatiales et temporelles concernant surtout la couche superficielle jusqu'\`{a} environ 100m.  En-dessous, la temp\'{e}rature moyenne reste stable. \newline \newline
L'\textbf{\underline{\'{e}quation d'\'{e}tat de l'eau de mer}} fournit la relation :}
\[c=1449,2+4,6T-0,055T^2+0,00029T^3+(1,34-0,01T)(S-35)+1,58\times10^{-6}p\]
\paragraph{\normalfont o\`{u} $c$ est la c\'{e}l\'{e}rit\'{e} en m.s\up{-1}, $T$ la temp\'{e}rature en \degre C , $S$ la salinit\'{e} en \textperthousand \space et $p$ la pression en Pa. La pression hydrostatique provoque une augmentation lin\'{e}aire de la c\'{e}l\'{e}rit\'{e} avec la profondeur d'environ 16m.s\up{-1} par kilom\`{e}tre. On peut donc remplacer le dernier terme par $0,016z$, o\`{u} $z$ est l'immersion en m. \newline \newline
\textbf{\underline{Profil bathyc\'{e}l\'{e}rim\'{e}trique et r\'{e}fraction :}} \newline \newline
Il est souvent possible de faire l'approximation d'une \textbf{stratification horizontale du milieu} (la variabilit\'{e} horizontale de la c\'{e}l\'{e}rit\'{e} est souvent n\'{e}gligeable) : \textbf{la c\'{e}l\'{e}rit\'{e} ne d\'{e}pend alors que de l'immersion}, et on appelle \textbf{profil bathyc\'{e}l\'{e}rim\'{e}trique} cette loi de d\'{e}pendance. Elle comporte une \textbf{strate homog\`{e}ne de c\'{e}l\'{e}rit\'{e} constante} (premiers m\`{e}tres d'immersion), une \textbf{isotherme} (\`{a} temp\'{e}rature constante, la c\'{e}l\'{e}rit\'{e} augmente lin\'{e}airement \`{a} cause de la pression hydrostatique), un \textbf{chenal de surface} (augmentation de la c\'{e}l\'{e}rit\'{e} depuis la surface \`{a} cause d'une couche superficielle isotherme), une \textbf{thermocline} (d\'{e}croissance monotone de la temp\'{e}rature avec l'immersion qui provoque une diminution de la c\'{e}l\'{e}rit\'{e}), et un \textbf{chenal profond} (minimum de c\'{e}l\'{e}rit\'{e}).}
\paragraph{\normalfont Consid\'{e}rons l'interface entre deux milieux fluides homog\`{e}nes o\`{u} les c\'{e}l\'{e}rit\'{e}s des ondes sont diff\'{e}rentes, not\'{e}es $c_1$ et $c_2$. Lors du changement de milieu, le changement de c\'{e}l\'{e}rit\'{e} provoque une r\'{e}flexion sp\'{e}culaire de l'onde dans le premier milieu, et une r\'{e}fraction de l'onde dans le second milieu, selon un angle donn\'{e} par la \textbf{\underline{loi de Snell-Descartes :}}}
\begin{figure}[!h]
\includegraphics[scale=0.75]{TIPE_FIG3.png}
\caption{Structure d'un profil bathyc\'{e}l\'{e}rim\'{e}trique en Atlantique Nord-Est. Ces profils peuvent beaucoup varier en fonction des zones g\'{e}ographiques et des conditions locales, par exemple aux hautes latitudes avec la fonte des glaces, ou encore au niveau du d\'{e}troit de Gibraltar o\`{u} les eaux chaudes M\'{e}diterran\'{e}ennes rejoignent celles plus froides de l'Atlantique.}
\label{fig_3}
\end{figure}
\[\frac{sin\theta_1}{c_1}=\frac{sin\theta_2}{c_2}\]
\paragraph{\normalfont Cette relation n'existe que si $sin\theta_2 \leq 1$ donc si $sin\theta_1 \leq c_1/c_2$. Si $\theta > arcsin (c_1/c_2)$ (angle critique de l'interface), il y a r\'{e}flexion totale et la transmission dans le second milieu est impossible.}
\paragraph{\normalfont En appliquant la loi de Snell-Descartes \`{a} un empilement de $n$ strates isoc\'{e}l\`{e}res d'indices $i = 1,2,...n$ , on obtient la relation:}
\begin{figure}[!h]
\includegraphics[scale=1]{TIPE_FIG4.png}
\caption{R\'{e}flexion et r\'{e}fraction d'une onde plane par un changement de c\'{e}l\'{e}rit\'{e} \`{a} une interface.}
\label{fig_4}
\end{figure}
\[\frac{sin\theta_i}{c_i}=\frac{sin\theta_{i+1}}{c_{i+1}}\]
\paragraph{\normalfont Donc pour une onde se propageant dans un milieu \`{a} c\'{e}l\'{e}rit\'{e} non constante selon une coordonn\'{e}e z, la relation de r\'{e}fraction donne :}
\[\frac{sin\theta(z)}{c(z)}=cte\]
\paragraph{\normalfont Par cons\'{e}quent,un chagement progressif de la c\'{e}l\'{e}rit\'{e} du milieu de propagation provoque une d\'{e}viation de la direction intiale de l'onde. Si le gradient de c\'{e}l\'{e}rit\'{e} est vertical, une augmentation de la c\'{e}l\'{e}rit\'{e} tend \`{a} r\'{e}fracter l'onde vers l'horizontale, tandis qu'une diminution de la c\'{e}l\'{e}rit\'{e} tend \`{a} augmenter l'angle de rasance de l'onde. Dans une strate affect\'{e}e d'un gradient lin\'{e}aire de c\'{e}l\'{e}rit\'{e} $g=dc/dz$, un rayon d'angle d'incidence $\theta$ \`{a} l'entr\'{e}e dans la strate subit une r\'{e}fraction de trajectoire en arc-de-cercle, dont le rayon de courbure est donn\'{e} par $\rho_c=c/gsin\theta$ o\`{u} $c$ est la c\'{e}l\'{e}rit\'{e} au point de la trajectoire correspondant \`{a} l'angle $\theta$.}
\paragraph{\normalfont \textbf{\underline{R\'{e}sultats exp\'{e}rimentaux :}} \newline
Nous avons \'{e}tudi\'{e} l'influence de la temp\'{e}rature sur la c\'{e}l\'{e}rit\'{e} des ondes en effectuant des mesures de longueurs d'ondes \`{a} diff\'{e}rentes temp\'{e}ratures, les autres param\`{e}tres \'{e}tant fix\'{e}s. Pour ce faire, nous avons visualis\'{e} \`{a} l'oscilloscope les signaux provenant d'un \'{e}metteur et d'un r\'{e}cepteur d'ultrasons (fr\'{e}quence 32kHZ) et effectu\'{e} des mesures de longueur d'onde (permettant de calculer la c\'{e}l\'{e}rit\'{e}) \`{a} diff\'{e}rentes temp\'{e}ratures. Les r\'{e}sultats obtenus sont les suivants :}
\begin{figure}[!h]
\includegraphics[scale=0.65]{TIPE_FIG5.png}
\caption{Gradient de c\'{e}l\'{e}rit\'{e} et trajet d'un rayon d'ondes correspondant (mod\`{e}le discret \`{a} gauche, lin\'{e}aire \`{a} droite)}
\label{fig_5}
\end{figure}
\paragraph{\normalfont Les valeurs de c\'{e}l\'{e}rit\'{e} obtenues par mesure sont plus \'{e}lev\'{e}es par que valeurs standards, qui se situent aux alentours de 1500m.s\up{-1}. N\'{e}anmoins, l'augmentation lin\'{e}aire de la c\'{e}l\'{e}rit\'{e} avec la temp\'{e}rature, toutes choses \'{e}gales par ailleurs, est v\'{e}rifi\'{e}e. Dans l'oc\'{e}an, la c\'{e}l\'{e}rit\'{e} des ondes augmente d'environ 4,5m.s\up{-1} par degr\'{e}. Nous avons observ\'{e} une augmentation de 5,1m.s\up{-1} par degr\'{e}. Les \'{e}carts aux valeurs attendues s'expliquent par les incertitudes et difficult\'{e}s de mesure, ainsi que par le fait que nous avons utilis\'{e} de l'eau douce et non sal\'{e}e.}
\begin{figure}[!h]
\includegraphics[scale=1]{TIPE_FIG6.png}
\caption{C\'{e}l\'{e}rit\'{e} en fonction de la temp\'{e}rature}
\label{fig_6}
\end{figure} \newpage
\paragraph{\normalfont \textbf{\underline{R\'{e}trodiffusion des ondes par une cible :}}\newline
L'onde acoustique est diffus\'{e}e par la cible dans toutes les directions de l'espace. Une partie de l'\'{e}nergie est notamment renvoy\'{e}e vers l'\'{e}metteur, donc \textbf{la cible agit comme une source secondaire qui r\'{e}\'{e}met l'onde acoustique}. L'\textbf{indice de cible IC} traduit le rapport en dB entre l'intensit\'{e} acoustique r\'{e}\'{e}mise par la cible vers la source et l'intensit\'{e} incidente selon la relation $IC=10log(I_r/I_i)$ o\`{u} $I_i$ est l'intensit\'{e} de l'onde incidente sur la cible (suppos\'{e}e localement plane) et $I_r$ l'intensit\'{e} de l'onde r\'{e}fl\'{e}chie r\'{e}fl\'{e}chie (suppos\'{e}e sph\'{e}rique \`{a} partir du centre acoustique de la cible). Cet indice traduit ainsi la quantit\'{e} d'\'{e}nergie acoustique renvoy\'{e}e par la cible vers le sonar, et d\'{e}pend de la nature de la cible (mat\'{e}riau, structure interne et externe), ainsi que du signal incident (angle et fr\'{e}quence). \newline \newline
Comme l'intensit\'{e} sonore varie comme l'inverse du carr\'{e} de la distance, cette grandeur physique pourrait \^{e}tre utilis\'{e}e pour d\'{e}terminer la distance entre un \'{e}metteur et une cible. N\'{e}anmoins, \textbf{\`{a} cause des ph\'{e}nom\`{e}nes de pertes d\'{e}crits plus haut, on privil\'{e}giera des mesures de d\'{e}calages temporels entre \'{e}mission et r\'{e}ception de signaux}. La c\'{e}l\'{e}rit\'{e} des ondes \'{e}tant connue en un lieu de pression, temp\'{e}rature, salinit\'{e} et profondeur donn\'{e}es, on peut d\'{e}duire de ces mesures la distance entre l'\'{e}metteur et la cible.} \newpage
\part{Une tentative de mod\'{e}lisation simplifi\'{e}e: mise en oeuvre d'un d\'{e}tecteur d'obstacles marins \`{a} ondes sonores}
\setcounter{section}{0}
\section{Hypoth\`{e}ses de travail}
\paragraph{\normalfont \textbf{\underline{Principe :}} \newline \newline Mod\'{e}lisation constituant en l'interpr\'{e}tation des r\'{e}sultats d'un \'{e}metteur/r\'{e}cepteur d'ondes acoustiques. \newline \newline
\textbf{\underline{D\'{e}finitions pr\'{e}liminaires :}} \newline \newline Il existe 2 principaux types de sonars sous-marins: les sonars \textbf{actifs} et les sonars \textbf{passifs.}\newline \newline
Les \textbf{sonars passifs} consistent en la simple \'{e}coute du bruit environnant sans en \'{e}mettre. Ils sont majoritairement utilis\'{e}s dans un cadre scientifique ou militaire. Ils sont beaucoup utilis\'{e} dans les sous-marins, puisqu'ils poss\`{e}dent une grande distance d'\'{e}coute, portant sur des centaines de kilom\`{e}tres, tout en assurant la discr\'{e}tion du vaisseau qui n'\'{e}met pas de bruit. Le sous-marins en patrouille \'{e}coute:\newline
- les sons \'{e}mis par les h\'{e}lices des autres bateaux, les coques en mouvement...  \newline
- le son \'{e}mis par les autres sonars\newline
- ses propres bruits, qu'il doit r\'{e}duire au maximum afin de rester discret\newline \newline
Les \textbf{sonars actifs} \'{e}mettent quant \`{a} eux des signaux et \'{e}coutent leur \'{e}chos sur les obstacles qu'ils rencontrent.\newline \newline
Le capteur \'{e}tudi\'{e} ci-apr\`{e}s \textbf{fonctionnera sur le mod\`{e}le du sonar actif}. \newline \newline
\textbf{\underline{Pourquoi ce choix ?}} \newline On peut ne pas vouloir d\'{e}tecter que des objets \'{e}mettant un son, mais \textbf{aussi et surtout} des obstacles, qui eux n'\'{e}mettent aucun son, les rendant ind\'{e}tectables au sonar passif. \`{A} noter que ce syst\`{e}me est inadapt\'{e} au domaine militaire \`{a} cause de la \textbf{rep\'{e}rabilit\'{e}} qu'il implique. \newline \newline
Nous allons donc calculer le retard entre l'onde \'{e}mise et l'onde r\'{e}fl\'{e}chie que l'on appellera "onde retour" pour estimer la distance \`{a} laquelle se trouve l'obstacle. Pour se donner une id\'{e}e du \textbf{mouvement de l'obstacle} (rapprochement, \'{e}loignement), on devra calculer la variation de fr\'{e}quence entre l'onde \'{e}mise et celle de retour et on pourra ainsi exploiter l'\textbf{effet Doppler}. \newline \newline
\textbf{\underline{Point th\'{e}orique:}}\newline\textbf{\underline{Effet Doppler-Fizeau :}} il consiste en une variation de la fr\'{e}quence d'une onde lorsque l'\'{e}metteur ou le r\'{e}cepteur s'approchent ou s'\'{e}loignent. Ici, l'\'{e}metteur sera l'obstacle et le r\'{e}cepteur sera le capteur du Sonar.}
\paragraph{\normalfont \textbf{\underline{Th\'{e}or\`{e}me :}}En r\'{e}f\'{e}rentiel galil\'{e}en:}
\begin{figure}[!h]
\includegraphics[scale=1]{TIPE_FIG7.png}
\caption{Repr\'{e}sentation qualitative du ph\'{e}nom\`{e}ne}
\label{fig_7}
\end{figure}
\[f_{rec}=\frac{c-v_{rec}}{c-v_{em}}\times f_{em}=\frac{1-(v_{rec}/c)}{1-(v_{em}/c}\times f_{em}\]
\paragraph{\normalfont avec $f_{rec}$: fr\'{e}quence de l'onde re\c{c}ue, \newline $f_{em}$: fr\'{e}quence de l'onde \'{e}mise, \newline $v_{rec}$: vitesse du r\'{e}cepeteur, \newline $v_{em}$: vitesse de l'\'{e}metteur,\newline $c$: c\'{e}l\'{e}rit\'{e} des ondes acoustiques aquatiques \`{a} la position spatiale et temporelle \'{e}tudi\'{e}e (voir Premi\`{e}re Partie)\newline \newline
\underline{Preuve:} Consid\'{e}rons un \'{e}metteur allant g\'{e}n\'{e}rer des impulsions de fr\'{e}quence $f_{em}$ (avec pour p\'{e}riode $T_{em}=1/f_{em}$), et que le mouvement relatif \'{e}metteur-r\'{e}cepteur se fasse selon la droite les joignant. \newline
Lors de la deuxi\`{e}me impulsion, la premi\`{e}re a parcouru la distance : $d_0=c.T_{em}$. Pendant ce temps, la source se sera d\'{e}plac\'{e}e de $\Delta=v_{em}.T_{em}$. On notera ainsi $d_1$ la distance s\'{e}parant les deux impulsions :}
\[d_1=d_0-\Delta \Rightarrow d_1=(c-v_{em}).T_{em}\]
\paragraph{\normalfont Il faut maintenant calculer $T_{rec}$, le temps s\'{e}parant la r\'{e}ception des deux impulsions par le r\'{e}cepteur. Pendant cette dur\'{e}e, le r\'{e}cepteur s'est d\'{e}plac\'{e} de $\delta=v_{rec}.T_{rec}$. Pendant ce temps $T_{rec}$, la seconde impulsion aura parcouru la distance $d_2=c.T_{rec}=d_1+v_{rec}.T_{rec}$. (distance entre les deux ondes \`{a} laquelle il faut ajouter le d\'{e}placement du r\'{e}cepteur). On a donc bien:}
\[f_{rec}=\frac{1}{T_{rec}}=\frac{c-v_{rec}}{d_1}=\frac{c-v_{rec}}{c-v_{em}}.\frac{1}{T_{em}}=\frac{c-v_{rec}}{c-v_{em}}.f_{em}\]
\paragraph{\normalfont \textbf{\underline{Cons\'{e}quence pour nous ici:}} on va vouloir d\'{e}terminer $v_{rec}$. Nous nous pla\c{c}erons de plus dans des conditions de mesures immobiles donc on aura $v_{em}$=0 m.s\up{-1}. On en d\'{e}duit donc la formule suivante, qui sera utilis\'{e}e dans l'algorithme:}
\[\boxed{v_{rec}=c(1-\frac{f_{rec}}{f_{em}})}\]
\paragraph{\normalfont La zone thermocline dont il a \'{e}t\'{e} question dans la premi\`{e}re partie pose de \textbf{gros probl\`{e}mes de mesure}, tant les variations de c\'{e}l\'{e}rit\'{e} sont importantes lorsqu'on envoie des ondes verticalement (il y a en revanche moins de soucis horizontalement). On peut donc, pour pallier \`{a} cela, \textbf{d\'{e}rouler un c\^{a}ble} au bout duquel est accroch\'{e} au capteur pour passer au-del\`{a} de la zone.  \textbf{Il sera n\'{e}cessaire de conna\^{i}tre la profondeur du capteur}. (Mise en \oe{}uvre possible par une large vari\'{e}t\'{e} de proc\'{e}d\'{e}s. Exemple: codes r\'{e}partis \`{a} intervalles r\'{e}guliers grav\'{e}s sur le c\^{a}ble et ensuite scann\'{e}s pour d\'{e}terminer la profondeur). On \textbf{supposera} ici que cette information est renvoy\'{e}e en brut par le capteur, sans s'int\'{e}resser \`{a} la lecture de codes grav\'{e}s. \newline \newline
Suivant cette profondeur et la mesure de temp\'{e}rature de l'eau par un thermom\`{e}tre embarqu\'{e}, on \textbf{d\'{e}termine une approximation de la c\'{e}l\'{e}rit\'{e} de l'eau dans les conditions de mesure}, en supposant la c\'{e}l\'{e}rit\'{e} de r\'{e}f\'{e}rence \`{a} 1500m.s\up{-1} \`{a} la surface, \`{a} 300K\protect\footnote{Source : \textit{Introduction et validation d'un mod\`{e}le de changement de phase dans le code de dynamique rapide Europlexus} (E. Blaud, P.Gallon, F. Daude), ResearchGate. \`{A} noter : la valeur 300K a \'{e}t\'{e} extraite de cette publication \`{a} titre indicatif, car la c\'{e}l\'{e}rit\'{e} des ondes acoustiques \textbf{pourra \^{e}tre diff\'{e}rente \`{a} m\^{e}me temp\'{e}rature mais salinit\'{e}s diff\'{e}rentes}, par exemple.}, soit 27°C. Cela a une importance pour affiner les mesures car sur de telles c\'{e}l\'{e}rit\'{e}s (env. 1500m.s{-1}) les variations de c\'{e}l\'{e}rit\'{e}, bien que petites, peuvent jouer dans la pr\'{e}cision de la distance calcul\'{e}e. Cela permettra d'optimiser le mod\`{e}le, mais il ne pourra que difficilement \^{e}tre parfait car on ne tiendra compte \textbf{ni de la composition exacte de l'eau au point de mesure}, \textbf{ni de l'agitation de l'oc\'{e}an (vagues et courants)}, ayant tous deux une influence sur la c\'{e}l\'{e}rit\'{e}. On ne peut en effet pas \'{e}tudier pour des questions de rapidit\'{e} \textbf{la composition de l'eau en chaque point}, et prendre en compte l'agitation de l'eau est bien trop complexe car un tel ph\'{e}nom\`{e}ne est localis\'{e} et une part d'al\'{e}atoire le r\'{e}git (m\^{e}me si l'on peut aujourd'hui pr\'{e}voir les courants et l'\'{e}tat de la mer, il est \`{a} noter par exemple que de petites vagues peuvent se glisser au milieu de plus grosses vagues, rendant toute exploitation absurde). \newline
\`{A} cet effet, l'utilisation du dispositif en surface en cas de mer agit\'{e}e ne sera pas trait\'{e}e car les creux des vagues fausseront toutes les mesures, qui renverront des r\'{e}sultats aberrants. \newline \newline
\textbf{\underline{Donn\'{e}es exp\'{e}rimentales: variations de la c\'{e}lerit\'{e}}:}}
\[\partial T= 1K \Rightarrow (\partial c)_T=3m.s^{-1}\]
\[\partial z= 100m \Rightarrow (\partial c)_z=1,7m.s^{-1}\]
\paragraph{\normalfont Pour \'{e}viter de capter le signal \'{e}mis directement lorsqu'il est \'{e}mis, ce qui laisserait peu d'int\'{e}r\^{e}t \`{a} la mod\'{e}lisation, on \textbf{\'{e}teint le capteur un court instant d\`{e}s le d\'{e}but d'\'{e}mission de l'impulsion sonore et on le remet en fonctionnement peu apr\`{e}s la fin d'\'{e}mission}. La dur\'{e}e de l'impulsion et de pause du r\'{e}cepteur en pratique seront d\'{e}finies et choisies plus loin. \newline \newline
\textbf{\underline{Se pose maintenant la question des pertes dans l'eau}}:}
\paragraph{\normalfont En travaillant \`{a} partir des mod\`{e}les de Thorp, Leroy, Fran\c{c}ois et Garisson, on remarque que l'on a fort int\'{e}r\^{e}t \`{a} \'{e}mettre une impulsion de basse fr\'{e}quence et aucunement un ultrason. En pratique on choisira \textbf{3Khz pour l'\'{e}mission}, et ainsi l'amortissement de l'onde \'{e}mise sera tr\`{e}s faible, permettant d'obtenir une large port\'{e}e\protect\footnote{que l'on bridera cependant en pratique} tout en \'{e}mettant des sons de puissance moyenne donc respectueux de la faune marine.}
\begin{figure}[!h]
\includegraphics[scale=0.8]{TIPE_FIG8.png}
\caption{Analyse de l'absorption d'\'{e}nergie par l'eau selon les mod\`{e}les de Thorp, Leroy, Fran\c{c}ois et Garisson}
\label{fig_8}
\end{figure}
\section{Proc\'{e}dure de traitement du signal}
\paragraph{\normalfont \textbf{\underline{Dur\'{e}e d'extinction capteur juste apr\`{e}s l'\'{e}mission :}}\newline
\textbf{\underline{Rappel du probl\`{e}me}} : Il faut arr\^{e}ter obligatoirement le r\'{e}cepteur un court instant pour ne pas recevoir l'echo de l'onde venant d'\^{e}tre \'{e}mise. Cela dit, \textbf{il faut aussi pouvoir d\'{e}tecter des objets proches et donc \'{e}viter que cette dur\'{e}e de "pause" ne soit trop longue}. \newline \newline
On fixe la \textbf{distance minimale de l'obstacle : 50 m}. \newline
On choisira d'\'{e}mettre une \textbf{impulsion de 55ms} (la dur\'{e}e de pause du capteur \`{a} partir du d\'{e}but de l'\'{e}mission sera de 66ms, donc les mesures reprendront 11ms apr\`{e}s la fin de l'impulsion, ce qui permettra de d\'{e}tecter la \textbf{pr\'{e}sence d'obstacles distants d'au moins 50m} pour un aller-retour de l'onde, puisqu'\`{a} 1500m.s\up{-1}, elle parcourt 100m en 66ms. \newline \newline
Le \textbf{\underline{Th\'{e}or\`{e}me de Nyquist-Shannon}} (1949) impose que la fr\'{e}quence d'\'{e}chantillonnage d'un signal soit sup\'{e}rieure au double de la fr\'{e}quence de ce signal pour que la d\'{e}termination de son spectre ait un sens. Nous \'{e}mettons \`{a} 3kHz mais la fr\'{e}quence du signal retour peut varier avec l'effet Doppler. Nous choisirons donc de ne traiter informatiquement les \'{e}chantillons que si ceux-ci ont une \textbf{fr\'{e}quence d'\'{e}chantillonnage sup\'{e}rieure \`{a} 7kHz}. \newline \newline
Autre n\'{e}cessit\'{e} : il faut que la \textbf{dur\'{e}e d'\'{e}chantillonnage soit grande devant la p\'{e}riode du signal} : l'impulsion \'{e}mise durant 55ms, on peut consid\'{e}rer que l'acquisition de l'impulsion retour aura environ la m\^{e}me dur\'{e}e. Pour une fr\'{e}quence retour comprise entre 2500 et 3500 Hz, on aura une p\'{e}riode pouvant se rapprocher de 0,5ms $<<$ 55ms \newline $\Rightarrow$ donc \underline{une mesure de 55ms convient}.}
\section{Un algorithme d'analyse du signal retour}
\paragraph{\normalfont Au vu des hypoth\`{e}ses venant d'\^{e}tre faites, et \`{a} l'aide de certains outils int\'{e}gr\'{e}s au module scipy de Python, comme la transform\'{e}e de Fourier Rapide, on peut proposer un algorithme d'analyse du signal retour comme suit:}
\paragraph{\normalfont}
\lstset{language=Python}
\begin{lstlisting}[mathescape]
import numpy as np
import numpy.fft as fft

##Chargement du fichier de l'acquisition
alpha, theta, temperature, deepness, noise_calib = np.loadtxt("G:\TIPE\\configuration.txt", delimiter=",",unpack=True) #on traite les donnees de configuration: noise_calib est un parametre de calibration du capteur qui va nous indiquer un seuil correspondant au bruit ambiant de l'ocean afin de reconnaitre le retour de l'impulsion emise lorsqu'on depassera ce seuil.
time, ampli = np.loadtxt("G:\TIPE\\acquisition.txt", delimiter=",",unpack=True) #on ouvre le signal retour

##Determination des variables allant etre utiles
duration=time[-1] #duree de l'acquisition
Te = time[1]-time[0] #on determine simplement la periode d'echantillonage du capteur
fe = 8000 #on en deduit la frequence correspondante
celerite = 1500 + (temperature-27)*3 + deepness*0.017 #celerite du son dans l'eau dans les conditions de mesure
print(abs(time))

##Calcul de la distance de l'obstacle
if fe<7000:
    autorisationFourier=False
    print("Erreur. La frequence d'echantillonage du capteur n'est pas assez elevee. Connectez un autre capteur.")
else:
    print("Parametres: Frequence d'echantillonage:", fe, "Hz.")
    for k in range(len(time)):
        if ampli[k]>noise_calib:
            retard=0.055+0.066+time[k] #on calcule le retard de l'onde en prenant en compte le temps depuis le demarrage du capteur ajoute au temps ayant separe le debut de l'emission
            print("L'obstacle est a une distance de",(retard*celerite)/2,"m sur l'orientation alpha =", alpha, "${^\circ}$ theta =",theta,"${^\circ}$.") #INFO I/ DISTANCE DE L'OBSTACLE
            if k+(int(0.055/Te))<len(time):
                time=time[k:k+(int(0.055/Te))] #on nettoie le signal retour pour ne garder que la partie        interessante
                ampli=ampli[k:k+(int(0.055/Te))]
                autorisationFourier=True
            else: #si on a pas capte l'onde retour en entier
                print("La vitesse de l'obstacle n'a pas pu etre determinee car la duree d'acquisition etait trop courte.")
                autorisationFourier=False
            break
        if k==len(time)-1:
            print("Pas d'obstacle dans la zone de portee du capteur sur l'orientation alpha =", alpha, "${^\circ}$ theta =",theta,"${^\circ}$")
            autorisationFourier=False 
            
##Analyse de Fourier
    if autorisationFourier==True:
        spectrum=fft.fft(ampli)
        freq = fft.fftfreq(len(spectrum)) #frequences en unites de frequence d'echantillonage (ceci est du au fonctionnement du module fft)
        mask = abs(spectrum) > 0.9*max(abs(spectrum)) #ceci est un masque de l'array contenant le spectre en frequence du signal qui ne conserve que les zones ou le spectre est proche de sa valeur maximale
        fretour = freq[max] #on applique ce masque a l'axe des frequences
        for k in fretour:
            if k > 0: #on prend la premiere valeur de k STRICTEMENT positive (fondamental)
                fr_unit=k #la fft produisant un symetrique en frequences negatives du spectre, on recupere seulement la valeur positive de la frequence correspondant au max d'amplitude
                break
        fr=fr_unit*fe #pour rappel l'axe freq etait en unites de frequence d'echantillonage
        speed = celerite*(1-(fr/3000)) #on calcul par effet Doppler la vitesse de l'obstacle
        print("Estimation de la vitesse de l'obstacle:", speed, "m/s")
\end{lstlisting} \newpage
\part{Approfondissements}
\setcounter{section}{0}
\section{Retour sur le ph\'{e}nom\`{e}ne de r\'{e}flexion}
\paragraph{\normalfont Nous avons vu que le milieu marin influence grandement la propagation des ondes sonores.\newline
Montrons que le ph\'{e}nom\`{e}ne de r\'{e}flexion peut \^{e}tre consid\'{e}r\'{e} comme total au niveau du dioptre air/eau. Pour cela, consid\'{e}rons une onde sonore sinusoidale plane arrivant sur le dioptre avec une incidence normale.}
\paragraph{\normalfont On prend $x=0$ au niveau du dioptre, et on d\'{e}finit un vecteur unitaire $\vec{u_x}$ le long de l'axe selon x. On sait que le changement de milieu donne lieu \`{a} une onde r\'{e}fract\'{e}e et une onde r\'{e}fl\'{e}chie.\newline
On d\'{e}finit l'\textbf{imp\'{e}dance acoustique d'un milieu} pour une pulsation fix\'{e}e: $Z=\frac{p}{v} $ avec p la pression et v la vitesse. On  note respectivement $Z_1$ et $Z_2$ l'imp\'{e}dance des deux milieux, $c_1$ et $c_2$ la vitesse de l'onde \`{a} travers ceux-ci et enfin $p_i$, $p_t$ et $p_r$ les diff\'{e}rentes pressions.\newline
On a : $v_i=\frac{1}{Z_1}\times p_i \times  \vec{u_x}$ ,
$v_r=-\frac{1}{Z_2}\times p_t \times  \vec{u_x}$ ,
et $v_t=\frac{1}{Z_2}\times p_t \times  \vec{u_x}$\newline
On d\'{e}finit alors $t_p=\frac{p_t}{p_i}$ et $r_p=\frac{p_r}{p_i}$ respectivement les coefficients de transmission et de r\'{e}flexion au niveau du dioptre.\newline
\textbf{On consid\`{e}re la pression totale et la vitesse totale continues au niveau de cette surface}.\newline \newline}
\begin{figure}[!h]
\includegraphics[scale=0.55]{TIPE_FIG9.png}
\caption{Notations et positionnement du probl\`{e}me}
\label{fig_9}
\end{figure}
\[ p_t(0,t)=p_i(0,t)+p_r(0,t)\]
\[\Rightarrow \frac{1}{Z_2}\times p_t(0,t)=\frac{1}{Z_1}\times p_i(0,t) - \frac{1}{Z_2}\times p_t(0,t) \]
\[
  \Rightarrow \left\{
      \begin{aligned}
        t_p=1 + r_p\\
       \frac{1}{Z_2}\times t_p=\frac{1}{Z_1}\times (1-r_p) \\
      \end{aligned}
    \right.
\]
\[
  \Rightarrow \left\{
      \begin{aligned}
        t_p=\frac{2Z_2}{Z_1+Z_2}\\
       r_p=\frac{Z_2-Z_1}{Z_1+Z_2} \\
      \end{aligned}
    \right.
\]
\paragraph{\normalfont \`{A} 25\degre C, l'imp\'{e}dance acoustique de l'air est d'environ 430 Pa.s/m et celle de l'eau est d'environ $1.5\times10^6$ Pa.s/m.\newline
On trouve $t_p\simeq5.7.10^{-4}$ et $r_p\simeq-0,99$ (l'onde allant de l'eau dans l'air).\newline
On peut donc affirmer que la quasi-totalit\'{e} du rayon incident est r\'{e}fl\'{e}chie.\newline
De m\^{e}me, comme l'imp\'{e}dance du sol est diff\'{e}rente de celle de l'eau, le \textbf{ph\'{e}nom\`{e}ne de r\'{e}flexion peut-\^{e}tre consid\'{e}r\'{e} comme quasi-total.}}
\section{Zones d'ombre et zones de convergence}
\paragraph{\normalfont Du fait du gradient de pression dans l'eau, on constate que \textbf{certaines zones ne sont jamais atteintes par les ondes \'{e}mises}, ce sont des \textbf{zones d'ombre}. Dans ces zones, les objets sont difficilement d\'{e}tectables directement. En effet, le seul moyen d'y acc\'{e}der est par la r\'{e}flexion sur le fond, ce qui est cause de grandes pertes \'{e}nerg\'{e}tiques. Ces zones sont \textbf{exploit\'{e}es par les sous-marins pour se cacher}.}
\begin{figure}[!h]
\includegraphics[scale=0.40]{TIPE_FIG10.jpg}
\caption{Sch\'{e}ma illustratif des zones d'ombre et de convergence}
\label{fig_10}
\end{figure}
\paragraph{\normalfont Aussi, on constate des \textbf{zones de convergence}, qui forment des \textbf{chenaux de propagation}. Les chenaux acoustiques sont en g\'{e}n\'{e}ral rep\'{e}rables par des \textbf{minima de c\'{e}l\'{e}rit\'{e} des profils bathyc\'{e}l\'{e}rim\'{e}triques}. Ils consistent en une zone de forte concentration d'\'{e}nergie ondulatoire, obtenue par r\'{e}fractions successives des ondes sur deux gradients constituant le chenal. Gr\^{a}ce \`{a} ce ph\'{e}nom\`{e}ne, les \textbf{pertes g\'{e}om\'{e}triques sont limit\'{e}es}, et de \textbf{plus longues port\'{e}es peuvent \^{e}tre obtenues}.}
\section{Transducteurs sous-marins}
\paragraph{\normalfont Les \textbf{transducteurs \'{e}lectroacoustiques} effectuent la transformation d'\'{e}nergie \'{e}lectrique en \'{e}nergie acoustiques et inversement. Le transducteur, aussi appel\'{e} hydrophone, est g\'{e}n\'{e}ralement constitu\'{e} d'un \textbf{r\'{e}cepteur pi\'{e}zo\'{e}lectrique}, qui convertit une variation de pression en une variation de tension \'{e}lectrique. On utilise les c\'{e}ramiques piezo\'{e}lectriques autour de leur fr\'{e}quences de r\'{e}sonance pour obtenir le meilleur rendement possible, mais un compromis doit \^{e}tre fait avec la largeur de la bande passante, pour avoir un spectre le plus large possible en fr\'{e}quence. Enfin, les transducteurs sont tr\`{e}s souvent \textbf{directifs}, c'est \`{a} dire qu'ils ont une direction privil\'{e}gi\'{e}e, afin de \textbf{mieux rep\'{e}rer les sons dans l'eau}.}
\paragraph{\normalfont }
\begin{figure}[!h]
\includegraphics[scale=0.75]{TIPE_FIG11.png}
\caption{Un hydrophone pi\'{e}zo\'{e}lectrique}
\label{fig_11}
\end{figure}
\begin{figure}[!h]
\includegraphics[scale=0.70]{TIPE_FIG12.png}
\caption{Caract\'{e}ristique de la r\'{e}ponse fr\'{e}quentielle d'un transducteur en r\'{e}ception}
\label{fig_12}
\end{figure}
\section{Impulsion du signal}
\paragraph{\normalfont Enfin, diff\'{e}rentes techniques d'impulsion peuvent \^{e}tre utilis\'{e}es.\newline
La plus courante consiste \`{a} envoyer une \textbf{impulsion de fr\'{e}quence fixe} (ou encore \textbf{monochromatique}), \textbf{"ping"}, \`{a} l'image de la mod\'{e}lisation simplifi\'{e}e vue dans le \textbf{Deuxi\`{e}me partie}. On envoie une sinusoide de fr\'{e}quence f pendant une dur\'{e}e T, et on r\'{e}cup\`{e}re simplement le signal \`{a} l'aide d'un filtre passe bande, puis le signal subit une quadration avant d'\^{e}tre int\'{e}gr\'{e} pour obtenir la \textbf{position} de l'obstacle. Cette technique est la plus utilis\'{e}e gr\^{a}ce \`{a} sa \textbf{simplicit\'{e}}.\newline
Sinon, certains transducteurs utilisent une \textbf{impulsion modul\'{e}e en fr\'{e}quence}, aussi nomm\'{e}e \textbf{"chirp"}.  Lors de l'\'{e}mission, la fr\'{e}quence du signal augmente continuement, ce qui offre de plus grande possibilit\'{e}s de spectres de bandes passantes et une meilleure r\'{e}solution du signal. Aussi, le gain est meilleur et donc la qualit\'{e} globale est accrue. Cependant, ce signal est \textbf{plus difficile \`{a} traiter}.} \newpage
\part*{Conclusion}
\paragraph{\normalfont L'\'{e}tude des ondes acoustiques aquatiques et de leurs usages est tr\`{e}s complexe. Nous avons seulement dress\'{e} ici un portrait global des diff\'{e}rents param\`{e}tres \`{a} prendre en compte. Comme le montre la derni\`{e}re partie, les hypoth\`{e}ses ayant permis d'esquisser notre algorithme de traitement du signal sont tr\`{e}s limitatives et des algorithmes bien plus complets pourraient \^{e}tre envisag\'{e}s. La pl\'{e}thore d'\'{e}quations r\'{e}gissant le trajet des ondes acoustiques en milieu aquatique est difficile \`{a} int\'{e}grer dans un seul mod\`{e}le, tant il y a d'interd\'{e}pendances. Cela montre de plus que la recherche ne cessera probablement pas de si t\^{o}t dans ce domaine tant il est vaste et son \'{e}tude d\'{e}taill\'{e}e r\'{e}cente. \newline \newline
\textbf{"Mais comment peut-on, dans le pr\'{e}sent, d\'{e}cider de ce qui se passera pour la v\'{e}rit\'{e} dans l'avenir ? Nous faisons oeuvre de proph\`{e}tes sans en avoir le don."}\newline
\textit{Le Z\'{e}ro et l'Infini} (1940) de Arthur Koestler.}
\end{document}
