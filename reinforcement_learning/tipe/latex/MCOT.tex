\documentclass[12pt,a4paper]{extarticle}
\usepackage[a4paper,margin=20mm]{geometry}
\usepackage[french]{babel}
\usepackage[utf8]{inputenc}
\usepackage[T1]{fontenc}
\usepackage{lmodern}
\usepackage{amsmath}
\usepackage{amssymb}
\usepackage{mathrsfs}
\usepackage{fancybox}
\usepackage{multicol}
\usepackage{graphicx}
\usepackage{wrapfig}
\usepackage{textcomp}
\usepackage{xcolor}
\usepackage{listings}
\newcommand{\ud}{\mathrm{d}}
\newcommand{\hsp}{\hspace{20pt}}
\newcommand{\HRule}{\rule{\linewidth}{0.5mm}}
\renewcommand{\labelitemi}{\cdot}
\renewcommand{\labelitemii}{\cdot}
\renewcommand{\labelitemiii}{\diamond}
\renewcommand{\labelitemiv}{\ast}

\begin{document}

\begin{center}

\huge{ \bfseries Utilisation de l'intelligence artificielle dans la manœuvre autonome de bateau}
\end{center}

\section*{Positionnement thématique}

Intelligence Artificielle, Apprentissage automatique, Méthodes stochastiques

\section*{Mots-clefs}

\begin{multicols}{2}

Français:

\begin{itemize}

\item Manœuvre de bateau en autonomie
\item Apprentissage par Renforcement
\item 
\item Port marin
\item Simulation physique



\end{itemize}

Anglais:

\begin{itemize}

\item Autonomous boat operation
\item Seaport
\item Reinforcement Learning
\item Physical Simulation


\end{itemize}

\end{multicols}

\section*{Bibliographie commentée}

\begin{itemize}
\item Context
\item Besoin
\item State of the art
\item liste de nouvelles technologies candidates ayant du potentielles
\end{itemize}


\indent Le commerce maritime est aujourd'hui l'un des plus importants secteurs dans le transport
de marchandises. Il soulève de nombreux problème tant internationalement avec les différentes 
routes commerciales. Mais aussi nationalement avec la nécessité d'augmenter la taille et le 
trafics des grands ports. Cependant, ces grands ports face à l'augmentation de la taille et du 
nombre de bateaux doivent aider les capitaines à la manœuvrer dans le port pour limiter les risques.
La solutions couramment utilisé consiste à la formation de pilotes spécifiques à un port pour 
ensuite les envoyer sur les bateaux pour remplacer le capitaine dès l'entrée du port. 
Ce procédé est coûteux et demande un effort logistique important, c'est pourquoi depuis quelque années des 
entreprises tels que Volvo et Yanmar Global se propose à trouver une solution quant à l'automatisation de 
ces manœuvres. Pour cela, elles font appelle à des technologie de calculs de trajectoire 
et d'asservissements. Elle nécessite de nombreuses balises de positionnement sur le port puis l'installation de nombreux capteurs sur chaque bateaux pour leur
permettre de mieux réagir face à l'environnement proche. Les manœuvre s'effectue à 
très faible allure pour permettre le maximum de réactivité du bateau, facilitant ainsi le 
travail d'asservissement.
\indent Une nouvelle technologie non encore implémentée pour ce problème nous intéresse grandement.
Il s'agit de l'intelligence artifielle. Il existe des algorithmes d'apprentissage dit
supervisé, pour lesquels la machine apprend depuis une base de donnée de comportement que
l'on souhaite qu'elle reproduise. Dans le cas d'une prise de décision depuis une image,
le réseau neuronal convolutif est devenu incontournable. La branche de la discipline qui
nous intéresse particulièrement est l'apprentissage par renforcement. Au lieu de décrire 
précisément le comportement que doit adopté la machine, on lui donne un fonction 
d'évaluation de son comportement. On ne sait pas comment agir pour atteindre ces comportements mais on sait si
ils correspondent à ce que l'on attend ou non. Soit l'algorithme apprend à évaluer ces
actions pour un état donné (basé sur de la valuation), soit il apprend une polique de
fonctionnement c'est à dire une fonction qui à une situation associe une action (basé sur
des poliques). 
 
  Vieille de plus de 5000 ans, l’astronomie est un domaine scientifique vaste et fécond.
Aujourd’hui l’astronomie se sépare en deux catégories, bien qu’elles ne soient pas
totalement hermétiques l’une à l’autre : l’astrophysique et l’astronomie d’observation [1].
C’est cette dernière branche qui fera l’objet de mon étude. Le problème le plus développé
sera celui de la turbulence atmosphérique, qui reste un des plus difficiles à traiter.
En effet, l’atmosphère n’est pas neutre pour les ondes électromagnétiques qui la
traversent, et crée des phénomènes de réfraction et de déformation du front d’onde dues
aux inhomogénéités qu’elle comporte [2].
La turbulence atmosphérique est particulièrement problématique pour l’observation,
provoquant des écarts allant de 2 à 20 secondes d’arc ; c’est le seeing auquel on associe des
tailles caractéristiques qui dépendent de la longueur d’onde. Dans le visible, un seeing
correct est de l’ordre d’une seconde d’arc.
Cette turbulence, générée par des inhomogénéités dues à la composition de l’air, à
certains courants, est chaotique et est prépondérante sur les 20 premiers kilomètres
d’altitude et ses effets peuvent être perçus à l’œil nu, comme la scintillation par exemple
[3,4,7].
Dans le but d’augmenter la résolution des instruments d’observation situés sur terre
et d’accéder à des détails jusqu’ici inconnus, l’optique adaptative représente une grande
avancée pour l’observation des astres.
Initiée par la défense américaine durant la Guerre Froide et le projet de « guerre des
étoiles », la dé-classification de ces travaux permit à la science d’avancer.
L’objectif de cette méthode est de corriger en temps réel la déformation du front
d’onde reçu par le système d’observation après passage dans l’atmosphère [4]. La mise en
œuvre de cette méthode est technologiquement très exigeante : elle nécessite l’emploi
d’instruments d’une précision inouïe, notamment l’usage de miroirs déformables, mais aussi
d’une capacité de calcul importante.
Elle trouve une application essentielle en ophtalmologie, où elle permet de corriger
les défauts introduits par les différents milieux de l’œil à des fins chirurgicales [5].

Le principe de l’optique adaptative répond au schéma suivant : le signal est acquis
puis analysé par un analyseur de front d’onde. Un paramètre important lors de l’acquisition
est r0 qui est une taille caractéristique correspondant à la taille d’une division de front
d’onde [6].
Parmi ces analyseurs de front d’onde, la technique de l’interférométrie est
largement utilisée et permet de remonter directement à la déformation locale du front
d’onde. En 1972, elle permettait d’analyser jusqu’à 10 000 fronts d’onde par seconde.
Un autre type d’analyseur de front d’onde est le Shack-Hartmann, il fera partie de
l’objet de mon étude.
Ensuite, informatiquement, on calcule la déformation qu’il faut appliquer au miroir
pour corriger le front d’onde à l’aide de différentes méthodes (en pratique : polynômes de
Zernike par exemple, mais ceux-ci ne seront pas l’objet de l’étude) [6,7]. Le système
informatique envoie les instructions au système de déformation du miroir, qui peut alors
agir par le biais de vérins.
Il existe deux types de miroirs, les miroirs dits segmentés, et les miroirs continus
utilisés dorénavant. Les premiers ont l’avantage d’être plus facilement déformables mais
créent des phénomènes de diffraction que les miroirs continus ne provoquent pas.
Cependant l’obtention d’un signal de référence n’est pas aussi simple qu’il n’y paraît.
Utiliser une source naturelle quasi ponctuelle (étoile) assez lumineuse et proche de l’astre
observé pour tester l’atmosphère localement n’est pas toujours possible. C’est pourquoi les
physiciens ont recours à l’usage de lasers permettant d’exciter les atomes de sodium de la
haute atmosphère qui émettent alors en retour, créant ainsi des pseudo-étoiles dites guides
[8].
 
\section*{Problématique}

\indent yuafkyjhvafgryzufjakyzfy

\section*{Objectifs du TIPE}

\begin{itemize}

\item Comprendre la logique de l'apprentissage par renforcement
\item Programmer notre propre algorithme d'apprentissage par reforcement sans utiliser de librairies d'intelligence artificielle
\item Créer une simulation discrète d'un déplacement de bateau
\item Reussir à faire stationner un bateau dans un port

\end{itemize}

\section*{Bibliographie}
\begin{itemize}

\item Steve with the backpro
\item Yann Le Cun with the CNN
\item Value Based with David Silvers
\item Andrej Kaparthy

\end{itemize}

\end{document}
