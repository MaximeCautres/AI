\documentclass[12pt,a4paper]{extarticle}
\usepackage[a4paper,margin=25mm]{geometry}
\usepackage[french]{babel}\usepackage[french]{babel}
\frenchbsetup{StandardLists=true} % à inclure si on utilise \usepackage[french]{babel}
\usepackage{enumitem}
\usepackage[utf8]{inputenc}
\usepackage[T1]{fontenc}
\usepackage{lmodern}
\usepackage{amsmath}
\usepackage{amssymb}
\usepackage{mathrsfs}
\usepackage{fancybox}
\usepackage{multicol}
\usepackage{graphicx}
\usepackage{wrapfig}
\usepackage{textcomp}
\usepackage[svgnames]{xcolor}
\usepackage{color}
\usepackage{listings}
\newcommand{\ud}{\mathrm{d}}
\newcommand{\hsp}{\hspace{20pt}}
\newcommand{\HRule}{\rule{\linewidth}{0.5mm}}
\renewcommand{\labelitemi}{\cdot}
\renewcommand{\labelitemii}{\cdot}
\renewcommand{\labelitemiii}{\diamond}
\renewcommand{\labelitemiv}{\ast}


\begin{document}

\begin{center}

\huge{{\color{SteelBlue}  \bfseries Utilisation de l'intelligence artificielle dans la manœuvre autonome de bateau}}
\end{center}

\section*{{\color{SteelBlue} \bfseries Positionnement thématique}}

Intelligence Artificielle, Apprentissage automatique, Méthodes stochastiques

\section*{{\color{SteelBlue} Mots-clefs}}

\begin{multicols}{2}

Français:

\begin{itemize}[label=\textbullet]

\item Manœuvre de bateau en autonomie
\item Port marin
\item Apprentissage par Renforcement
\item Simulation physique



\end{itemize}

Anglais:

\begin{itemize}[label=\textbullet]

\item Autonomous boat operation
\item Seaport
\item Reinforcement Learning
\item Physical Simulation


\end{itemize}

\end{multicols}

\section*{\color{SteelBlue} {Bibliographie commentée}}

\indent Le commerce maritime est aujourd'hui l'un des plus importants secteurs dans le 
transport de marchandises. Il soulève de nombreux problèmes tant internationalement avec 
les différentes routes commerciales que nationalement avec la nécessité 
d'augmenter le trafic des grands ports. De plus, ces grands ports 
avec l'augmentation de la taille et du nombre de bateaux doivent aider les capitaines 
à manœuvrer dans le port pour limiter les risques de collisions et fluidifier le trafic.
La solution couramment utilisée consiste à la formation de pilotes spécialistes du port envoyés sur 
les bateaux pour remplacer les capitaines dès l'approche de la zone {\bfseries [1]}. Ce procédé est coûteux 
et demande un effort logistique important, c'est pourquoi depuis quelques années des 
entreprises telles que Volvo {\bfseries [2]} et Yanmar {\bfseries [3]} se proposent de trouver une solution quant à 
l'automatisation de ces manœuvres. Pour cela, elles font appel à des technologies de 
calculs de trajectoires et d'asservissement. Elles nécessitent l'installation de nombreuses balises de 
positionnement sur le port puis de nombreux capteurs sur chaque bateaux 
pour permettre aux bateaux de mieux réagir face à l'environnement. Les manœuvres 
s'effectuent à très faibles allures pour permettre le maximum de réactivité, 
facilitant ainsi le travail d'asservissement. \\
\indent Une nouvelle technologie non encore implémentée semble très prometteuse. 
Il s'agit de l'intelligence artificielle. La discipline naît dans les années
cinquante et le concept d'algorithme qui apprend est dès lors exposé par Alan Turing {\bfseries [4]}.
Le concept a mûri avec les progrès de l'informatique. Le premier grand succès est la 
victoire de DeepBlue contre Kasparov en 1998 {\bfseries [5]}, considéré comme le meilleur joueur d'échec 
de tout les temps. Cependant, l'algorithme n'avait rien d'intelligent. Aujourd'hui, 
grâce aux travaux de Yann LeCun, Geoffrey Hinton et d'autres {\bfseries [6]}, les méthodes ont été 
totalement bouleversées avec l'introduction de la notion d'apprentissage {\bfseries [7]}.
Les nouveaux algorithmes d'apprentissage sont de deux types. Les premiers sont qualifiés de supervisés {\bfseries [8]},
la machine apprend depuis une base de données des comportements que
l'on souhaite qu'elle reproduise. Ils reposent sur l'utilisation d'un réseaux 
de neurones. Dans le cas où l'on utilise des images, les réseaux de neurones convolutifs {\bfseries [9]}
sont devenus incontournables. Cette méthode est très efficace mais nécessite des bases 
de données labellisées très importantes, ce qui pour certaines tâches n'est pas accessible. 
Pour ces situations, une nouvelle approche dénommée apprentissage non supervisé {\bfseries [8]}, a permis
de faire de grand progrès. Une de ses branches semble particulièrement adaptée à 
l'automatisation des manœuvres de bateau, l'apprentissage par renforcement {\bfseries [10]}. Au lieu de 
décrire précisément le comportement que doit adopter la machine, on lui donne une fonction 
d'évaluation de celui-ci. On ne sait pas comment agir pour l'atteindre mais on sait s'il correspond à ce que l'on attend ou pas. Soit 
l'algorithme apprend à évaluer ces actions pour un état donné (fondé sur de la valuation) {\bfseries [11]},
soit il apprend une politique de fonctionnement c'est-à-dire une fonction qui à une 
situation associe une action (fondé sur des politiques) {\bfseries [12]}. Toutes ces techniques utilisent des 
algorithmes d'apprentissages, le plus connu étant la rétro-propagation {\bfseries [13]}. Cela permet de 
partir d'un réseau de neurones aléatoirement définis pour, au fur et à mesure des étapes 
d'entraînement, converger vers le cerveau qui répond le plus justement possible à la demande. Ces 
technologies d'apprentissage commencent à être utilisées pour la conduite autonome sur route {\bfseries [14]}
mais ne sont pas encore certifiées, par manque de compréhension de la technologie.

 
\section*{{\color{SteelBlue} Problématique}}

\indent Comment peut-on utiliser l'intelligence artificielle pour permettre à un bateau de manœuvrer dans un port dans le but de minimiser les dépenses liées à l'augmentation du trafic tout en garantissant la sécurité.  

\section*{{\color{SteelBlue} Objectifs du TIPE}}

\begin{itemize}[label=\textbullet]

\item Réussir à faire stationner un bateau dans un port
\item Programmer notre propre algorithme d'apprentissage par renforcement sans utiliser de librairie dédiée.
\item Créer une simulation discrète et réaliste d'un déplacement de bateau prenant en compte l'inertie et la viscosité.
\item Comprendre et réussir à manipuler les concepts sur lesquels sont basés l'intelligence artificielle.
\item Implémenter différentes technologies pour pouvoir comparer les performances et trouver la meilleure solution technique à notre problème.


\end{itemize}

\section*{{\color{SteelBlue} Bibliographie}}

\begin{enumerate}[label = {\bfseries [\arabic*] } ]

\item FFPM. Rôle des pilotes. Disponible sur http://public.pilotes-maritimes.com/le-pilotage-en-france/role-des-pilotes/ (Consulté en 09/2019).
\item Volvo Penta. Volvo Penta unveils pioneering selfdocking yacht technology. Disponible sur https://www.volvopenta.com/marineleisure/en-en/news/2018/jun/volvo-penta-unveils-pioneering-self-docking-yacht-technology.html (Consulté en 09/2019).
\item Yanmar. Yanmar Develops Basic Technology with JAMSTEC for Auto-navigation Robotic Boat and Auto-docking System. Disponible sur https://www.yanmar.com/us/news/2019/02/07/50938.html (Consulté en 09/2019).
\item TURING, Alan. Computing Machinery and Intelligence [en ligne]. 1950. Format PDF. Disponible sur: https://www.csee.umbc.edu/courses/471/papers/turing.pdf (Consulté en 10/2018).
\item PITTET, Lionel. Garry Kasparov: «Ma défaite contre Deep Blue était une victoire pour l’humanité», LE TEMPS [En ligne]. 2019. Disponible sur https://www.letemps.ch/sport/garry-kasparov-defaite-contre-deep-blue-etait-une-victoire-lhumanite (Consulté en 06/2019).
\item VINCENT, James. ‘Godfathers of AI’ honored with Turing Award, the Nobel Prize of computing, THE VERGE [En ligne]. 2019. Disponible sur https://www.theverge.com/2019/3/27/18280665/ai-godfathers-turing-award-2018-yoshua-bengio-geoffrey-hinton-yann-lecun (Consulté en 04/2019).
\item FRIDMAN, Lex. MIT Deep Learning [En ligne]. Disponible sur https://deeplearning.mit.edu/ (Consulté en 2019).
\item NIELSEN, Michael. Using neural nets to recognize handwritten digits [En ligne]. Disponible sur http://neuralnetworksanddeeplearning.com/chap1.html (Consulté en 2019)
\item LECUN, Yann and BENGIO, Yoshua. Convolutional Networks for images, speech, and Time-Series [En ligne]. Format PDF. Disponible sur http://yann.lecun.com/exdb/publis/pdf/lecun-bengio-95a.pdf (Consulté en 10/2019).
\item Reinforcement Learning, WIKIPEDIA [En ligne]. Disponible sur https://en.wikipedia.org/wiki/Reinforcement\textunderscore learning (Consulté en 09/2019).
\item David Silvers
\item KARPATHY Andrej, Deep Reinforcement Learning: Pong from Pixels [En ligne]. 2016. Disponible sur http://karpathy.github.io/2016/05/31/rl/ (Consulté en 2019).
\item NIELSEN, Michael. How the backpropagation algorithm works [En ligne]. Disponible sur http://neuralnetworksanddeeplearning.com/chap2.html (Consulté en 2019).
\item HOTZ, George Francis. comma.ai [En ligne].
Disponible sur https://comma.ai/vehicles (Consulté en 10/2019).
\end{enumerate}

\end{document}

